\chapter{Backends}

\section{MIR interpreter}

\sysname{} provides a backend that can run in a \commonlisp{} system,
either intrinsically or extrinsically.  This backend is mainly used
in order to test that generated MIR code yields the expected result. 

The MIR interpreter works as follows:

\begin{itemize}
\item The MIR code is translated to a \commonlisp{} program.  This
  program contains a lambda expression for each nested function.  The
  outermost lambda expression has no parameters.
\item The lambda expression is converted to a function using
  \texttt{compile}. 
\item The resulting function is executed by passing it to
  \texttt{funcall}. 
\end{itemize}

Each nested function has the following shape:

\begin{itemize}
\item The outermost form is a lambda expression with a lambda list of
  a particular form.  See below for details about the lambda list.
\item The body of the lambda expression is a \texttt{let} special
  form.  The variables of the \texttt{let} form are all the lexical
  variables used by the function, but not used by an enclosing
  function.  The \texttt{let} variables are all initialized to
  \texttt{nil}. 
\item The body of the \texttt{let} is a \texttt{block} special form.
  The name of the block is \texttt{nil}.  This block is used in the
  translation of the \texttt{return} instruction.  
\item The body of the block is a \texttt{tagbody} special form.  Each
  instruction generates a statement of the tagbody.  A statement may
  be preceded by a \texttt{tag} if there is a \texttt{go} to that
  statement. 
\end{itemize}

\begin{itemize}
\item The lambda list of the function has no initialization forms.
  An \texttt{\&optional} parameter is a list of two symbols, the
  parameter itself and a \emph{supplied-p} parameter.  The latter is
  included even when the original 
\end{itemize}

%%  LocalWords:  Backends backend
