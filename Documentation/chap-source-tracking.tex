\chapter{Source tracking}

We implement source tracking by having the function for converting
source named \texttt{cleavir-generate-ast:generate-ast} accept a
\emph{concrete syntax tree} (or \emph{CST} for short) as its first
argument, rather than just a source expression.

A CST has the exact same structure as the corresponding tree
representing the source expression, except that some of the nodes in
the tree have been augmented with information on source location.

We can describe the structure of the CST for some expression $E$ by
means of a transformation $C$ operating on $E$ as follows:

\begin{itemize}
\item $C[E] = CST[S, E, ()]$ if $E$ is an atom.
\item $C[E] = CST[S, E,
  (C[e_1]\enskip{}C[e_2]\enskip{}\cdots\enskip{}C[e_n])]$ if $E$ is
  the list $(e_1\enskip{}e_2\enskip{}\cdots\enskip{}e_n)$.
\item $C[E] = CST[S, E,
  (C[e_1]\enskip{}C[e_2]\enskip{}\cdots\enskip{}C[e_{n-1}]\enskip{}.\enskip{}C[e_n])]$
  if $E$ is the list
  $(e_1\enskip{}e_2\enskip{}\cdots\enskip{}e_{n-1}\enskip{}.\enskip{}e_n)$
  where $e_n$ is an atom other than \texttt{nil}.
\end{itemize}

In these equations, $CST$ is the constructor for CSTs, $S$ is the
source information relating to $E$.  As we can see, the constructor
for CSTs takes three arguments: the source information for the
expression, the expression itself and the list of transformed
\emph{children} of $E$.  If $E$ is a proper list, then the elements of
that list are considered the children of $E$.  If $E$ is a dotted
list, then the elements of the list and the \emph{atomic tail} of the
list are considered the children of $E$.

An implementation has basically two options for creating a CST:

\begin{itemize}
\item It can provide a special version of \texttt{read} that returns a
  CST rather than the raw expression.
\item It can take the raw expression returned by an ordinary
  \texttt{read} together with source information about the components
  of that expression and create a CST. 
\end{itemize}

For example, an implementation that maintains an \texttt{eq} hash
table mapping every \texttt{cons} of the raw expression to some source
location (but that does not provide source information for atoms) can
create a CST using the following technique:

{\small\begin{verbatim}
(defun make-cst (expression table)
  (make-instance 'cst
    :expression expression
    :source-info (gethash expression table)
    :children
    (if (atom expression)
        nil
        (loop with result = '()
              for remaining = expression then (rest remaining)
              until (atom remaining)
              do (push (make-cst (first remaining) table) result)
              finally (return (append (nreverse result)
                                      (if (null remaining)
                                          nil
                                          (make-cst remaining table))))))))
\end{verbatim}}

%% Source tracking is necessarily implementation-specific.  In this
%% chapter, we describe a protocol for source tracking that allows each
%% implementation to configure the protocol according to its methods and
%% abilities.

%% The function \texttt{cleavir-ast:generate-ast} takes an additional
%% optional argument named \texttt{source-info}.  The caller passes an
%% object that encapsulates source information about the form to be
%% compiled.

%% \sysname{} calls two generic functions:

%% \Defgeneric {begin-source} {source-info expression}

%% This function is called at the beginning of the processing of
%% \textit{expression}.  It should return an object representing the
%% source location for \textit{expression}, or \texttt{nil} if the source
%% location of \textit{expression} is not known.

%% \Defgeneric {end-source} {source-info}

%% This function is called at the end of the processing of the expression
%% that was most passed to the most recent call to
%% \texttt{begin-source}.

%% These functions are called in \emph{last-in first-out} order,
%% effectively defining a \emph{stack} of nested expressions.  The
%% implementation can keep track of that stack in order to have a better
%% idea of the source location of the expression, or it can ignore this
%% information and just use a hash table, mapping expressions to
%% locations.
