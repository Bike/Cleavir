\chapter{Environment}
\label{chap-environment}

\section{Introduction}

Compiling a \cl{} program is done in constant interaction with an
\emph{environment}.  The \hs{} stipulates%
\footnote{See Section 3.2.1 of the \hs{}.}
that there are four different environments that are relevant to
compilation:

\begin{itemize}
\item The \emph{startup environment}.  This environment is the global
  environment of the \cl{} system when the compiler was invoked.
\item The \emph{compilation environment}.  This environment is the
  local environment in which forms are compiled.  It is also the
  environment that is passed to macro expanders.
\item The \emph{evaluation environment}.  According to the \hs{}, this
  environment is ``a run-time environment in which macro expanders and
  code specified by \texttt{eval-when} to be evaluated are
  evaluated''. 
\item The \emph{run-time environment}.  This environment is used when
  the resulting compiled program is executed.
\end{itemize}

The \hs{} does not specify how environments are represented, and there
is no specified protocol for manipulating environments.  As a result,
each implementation has its own representation and its own protocols. 

It might seem that \sysname{} could represent the \emph{local part} of
the environment (i.e., the part of the environment that is temporarily
introduced when nested forms are compiled) in whatever way it pleases,
but this is not the case.  The reason is that the full environment
must be passed as an argument to macro expanders that are defined in
the startup environment, and those macro expanders are implementation
specific.  It is also not possible for \sysname{} to define its own
version of \texttt{macroexpand}, because a globally defined
implementation-specific macro expander may call the
implementation-specific version of \texttt{macroexpand} which would
fail if given an environment other than the one defined by the
implementation. 

On the other hand, an incomplete implementation for which no
representation of the local part of the environment has been
determined might want to use some such representation that is known to
work.  For that reason, \sysname{} proposes a \emph{default
  representation} of that local part of the environment.  The default
version can also be used by more complete implementations, provided
that the implementation-specific version of \texttt{macroexpand} is
not called on an environment using this default representation. 

\section{Querying the environment}

\subsection{Variable information}

\Defgeneric {variable-info} {environment name}

This function is called by the compiler whenever a symbol in a
\emph{variable} position is to be compiled.  It returns an instance of
one of the classes described below.

\Defclass {variable-info}

This class is the common superclass for lexical and special
variables. 

\Defmethod {type} {(info {\tt variable-info})}

Given an instance of the class \texttt{variable-info}, this method
returns the \emph{type} of the variable. 

\Defclass {lexical-variable-info}

This class is a subclass of the class \texttt{variable-info}.  It
represents information about lexical variables. 

\Defmethod {location} {(info {\tt lexical-variable-info})}

This method returns some kind of implementation-defined
\emph{location}.  The nature of the object returned is not specified,
but it must be the \emph{same} object for variables that are the
same. 

\Defclass {special-variable-info}

This class is a subclass of the class \texttt{variable-info}.  It
represents information about special variables. 

\Defmethod {name} {(info {\tt special-variable-info})}

This method returns the name of the special variable.

\Defclass {constant-info}

\Defmethod {value} {(info {\tt constant-info})}

\Defclass {symbol-macro-info}

\Defmethod {expansion} {(info {\tt symbol-macro-info})}

\subsection{Function information}

\subsection{Block information}

\Defgeneric {block-info} {environment name}

\Defclass {block-info}

\Defmethod {name} {(info {\tt block-info})}

\subsection{Tag information}

\Defgeneric {tag-info} {environment name}

\Defclass {tag-info}

\Defmethod {name} {(info {\tt tag-info})}

%%  LocalWords:  startup expanders expander
