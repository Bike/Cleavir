\chapter{Introduction}
\pagenumbering{arabic}%

\sysname{} is an implementation-independent framework for creating
\commonlisp{} compilers. 

It is implementation-independent in that it provides:

\begin{itemize}
\item general features that every implementation needs,
\item features that implementations can optionally choose to take
  advantage of, 
\item alternative features that are appropriate for some
  implementations and not for others,
\item mechanisms for allowing implementation-specific features that
  integrate seamlessly into the general framework.
\end{itemize}

To use the framework, an implementation provides methods for a few
generic functions that allow \sysname{} to access the compilation and
evaluation environments of the implementation.  \sysname{} uses those
functions to turn a form into an \emph{abstract syntax tree}, or AST.
\seechap{chap-abstract-syntax-tree} \sysname{} then translates the AST
into an intermediate representation called MIR.  \seechap{chap-mir}
This translation can be customized by the implementation.  \sysname{}
then provides a number of transformations on this representation such
as:

\begin{itemize}
\item translation into \emph{static single assignment} form,
\item type inference,
\item standard optimizations such as value numbering, common
  subexpression elimination, redundancy elimination, etc.
\end{itemize}

Once these implementation-independent and backend-independent
transformations have been accomplished, the MIR notation is gradually
transformed into a notation that is specific both to the
implementation and to the backend.  The notation becomes
backend-specific in that it introduces features such as registers and
stack frames.  The notation also becomes implementation specific in
that it exposes choices such as argument passing.

Finally, the low-level code is translated into machine code.

At each stage, an implementation can customize the process by
introducing new classes and methods. 
